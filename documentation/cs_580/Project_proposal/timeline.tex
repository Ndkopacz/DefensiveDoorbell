\section{Timeline}
\subsection*{Week 1 (10/19 - 10/25): Research Techniques for Model Analysis}
During the first week, we will focus on researching model interpretability and robustness techniques that we want to apply to analyze the machine learning model. 
\begin{itemize}
    \item Review literature on techniques such as LIME (Local Interpretable Model-agnostic Explanations) to explain model predictions.
    \item Explore robustness testing strategies, including noise injection, low-light simulation, and model perturbation testing.
    \item Identify performance benchmarks and real-time constraints relevant to our application on the Raspberry Pi.
\end{itemize}

\subsection*{Week 2 (10/26 - 11/1): Create Small Dataset (Red Hat Detection)}
The second week will involve creating a small dataset tailored for detecting individuals wearing a red hat.
\begin{itemize}
    \item Capture diverse images using the Raspberry Pi in real-world environments.
    \item Ensure the dataset includes variations in lighting, angles, and distances to improve generalization.
    \item Label data appropriately and split it into training and testing sets.
\end{itemize}

\subsection*{Week 3 (11/2 - 11/8): Train the Model (Fine-tuning YOLO)}
In the third week, we will fine-tune the YOLO model with the newly created dataset.
\begin{itemize}
    \item Load a pre-trained YOLO model and apply transfer learning techniques.
    \item Modify the model to classify a single output class (red hat detection).
    \item Ensure the model is optimized for both accuracy and speed, considering the limitations of the Raspberry Pi.
\end{itemize}

\subsection*{Week 4 (11/9 - 11/15): Test the Model Using Selected Techniques}
During this week, the trained model will be tested using the research techniques selected in Week 1.
\begin{itemize}
    \item Apply interpretability techniques (e.g., LIME) to analyze the model's decision-making process.
    \item Perform robustness testing by introducing noisy or perturbed data and measure the model's accuracy.
    \item Fine-tune hyperparameters based on results from the testing.
\end{itemize}

\subsection*{Week 5 (11/16 - 11/22): Port the Model to the Raspberry Pi and Test Speed}
Once the model is tested, we will port it to the Raspberry Pi to evaluate real-world performance.
\begin{itemize}
    \item Deploy the model on the Raspberry Pi and assess its inference speed.
    \item Test the model in real-time to ensure that it meets performance requirements.
    \item Optimize the model further to balance speed and accuracy.
\end{itemize}

\subsection*{Week 6 (11/23 - 11/29): Finalize Pipeline for Model Training and Testing}
This week focuses on finalizing the pipeline for training and testing models.
\begin{itemize}
    \item Document the entire training and testing process.
    \item Ensure reproducibility by packaging the code and defining clear steps for running the pipeline.
    \item Automate the pipeline for faster iteration on future model versions.
\end{itemize}

\subsection*{Week 7 (11/30 - 12/6): Create New Dataset with Perturbations}
In Week 7, we will create a new dataset that introduces various types of perturbations to test the model's robustness.
\begin{itemize}
    \item Simulate adverse conditions such as blurred images, low-light environments, and occluded objects.
    \item Label and prepare the dataset for model retraining.
\end{itemize}

\subsection*{Week 8 (12/7 - 12/8): Train a New Model and Send It Through the Pipeline}
During this week, we will train a new model on the perturbed dataset and test it using the previously established pipeline.
\begin{itemize}
    \item Fine-tune the model to handle the newly introduced perturbations.
    \item Run the updated model through the pipeline to evaluate performance, robustness, and interpretability.
\end{itemize}

\subsection*{Week 8 (12/7 - 12/8): Write and Submit the Paper}
Finally, we will compile the project findings and results into a formal research paper.
\begin{itemize}
    \item Write the final paper, summarizing our methodology, experiments, and results.
    \item Submit the paper by the deadline (12/8).
\end{itemize}